\documentclass[11pt, uplatex, dvipdfmx]{jsarticle}
\usepackage{graphicx}
\usepackage{amsmath,amsfonts,enumerate,bm,fancyhdr, braket, setspace, emathEy}
\usepackage{plext}
\newcommand{\ds}{\displaystyle}

\renewcommand{\tagform}[1]{(#1)}%
\preEqlabel{}

\pagestyle{fancy}


\chead{積分基本問題集} 

\begin{document}

\thispagestyle{empty}


\begin{center}
  \pbox<t>{\Huge \hspace{1.5in} 積 分 基 本 問 題 集 壱}
\end{center}

\newpage

次の積分または広義積分を求めよう.

  \vspace{.2in}

  \begin{edaenumerate}<retusuu=2, gyoukan=.3in>[(1)]
    
  \item $\ds \int_{-1}^{5} \left( 3x^3-5x-1 \right) \ dx$
    
  \item $\ds \int_{0}^{-2} \left( 3x-5 \right)^6 \ dx$
    
  \item $\ds \int_{1}^{4} \frac{dx}{(2x+1)^2}$
    
  \item $\ds \int_{0}^{\pi} \sin x \ dx$
    
  \item $\ds \int_{-\pi}^{\pi} \cos x \ dx$

  \item $\ds \int_{-\pi /6}^{\pi /4} \tan x \ dx$

  \item $\ds \int_{\pi/4}^{\pi/3} \frac{dx}{\tan x}$

  \item $\ds \int_{0}^{1} \sin^{-1}x \ dx$

  \item $\ds \int_{-1}^{1} \cos^{-1}x \ dx$

  \item $\ds \int_{-\sqrt{3}}^{1} \tan^{-1}x \ dx$

  \item $\ds \int_0^1 e^x \ dx$

  \item $\ds \int_{1/2}^{3} \log x \ dx$

  \item $\ds \int_{0}^{\pi} x \sin x \ dx$

  \item $\ds \int_{-\pi}^{\pi} \left(\sin x\right) \cos \left(2x \right)\ dx$

  \item $\ds \int_{-1}^{1} \frac{dx}{1+x^2}$

  \item $\ds \int_{0}^{2} \frac{dx}{4+x^2}$

  \item $\ds \int_{0}^{2} x^2 e^{-2x} \ dx$

  \item $\ds \int_{-1/2}^{1/2} \frac{dx}{\sqrt{1-x^2}}$

  \item $\ds \int_{-3/2}^{3/\sqrt{6}} \frac{dx}{\sqrt{3-x^2}}$

  \item $\ds \int_{0}^{\sqrt{2}} \frac{dx}{\sqrt{x^2+2}}$

  \item $\ds \int_{0}^{1} \sqrt{1-x^2} \ dx$

  \item $\ds \int_{0}^{2} \sqrt{1+4x^2} \ dx$

  \item $\ds \int_{0}^{1} 3^x \ dx$

  \item $\ds \int_{1}^{2} \log_2 x \ dx$

  \item $\ds \int_{0}^{1} \frac{x^3}{\sqrt{x^2+5}} \ dx$

  \item $\ds \int_1^e \left( \log x \right)^2 \ dx$

  \item $\ds \int_{0}^{1} \frac{dx}{(x^2+1)^2}$

  \item $\ds \int_{0}^{1} \frac{dx}{x^2+x+1}$

  \item $\ds \int_{-1}^{1} \frac{dx}{x^2-4}$

  \item $\ds \int_{0}^{1} \frac{x+2}{(x^2+x+1)^2} \ dx$

  \item $\ds \int_{1}^{2} \frac{dx}{x^2(x+1)}$

  \item $\ds \int_{0}^{\frac{1}{2}} \frac{1+x^2}{1-x^2} \ dx$

  \item $\ds \int_{0}^{1} \frac{dx}{x^4+1}$

  \item $\ds \int_{1}^{4} \frac{dx}{\sqrt{x}+1}$

  \item $\ds \int_{1}^{\sqrt{2}} \frac{dx}{x^3+x}$

  \item $\ds \int_{0}^{1} \frac{dx}{\sqrt{x}}$

  \item $\ds \int_{1}^{\infty} \frac{dx}{x^3}$

  \item $\ds \int_{0}^{2} \frac{dx}{\sqrt{4-x^2}}$

  \item $\ds \int_{1}^{2} \frac{dx}{x\sqrt{x-1}}$

  \item $\ds \int_{-1}^{1} \frac{dx}{\sqrt[4]{(1+x)^3}}$

  \item $\ds \int_{-2}^{2} \frac{x}{\sqrt{4-x^2}} \ dx$

  \item $\ds \int_{0}^{\infty} x e^{-x^2} \ dx$

  \item $\ds \int_{0}^{1} x \log x \ dx$

  \item $\ds \int_{-\infty}^{\infty} \frac{dx}{x^2+4}$

  \item $\ds \int_{1}^{\infty}\frac{\log x}{x^2} \ dx$

  \item $\ds \int_{0}^{\infty} \frac{dx}{e^x (1+e^x)}$

  \item $\ds \int_{0}^{\infty}\frac{dx}{(x^2+1)(x^2+4)}$

  \item $\ds \int_{0}^{1} \frac{x^4}{\sqrt{1-x^2}} \ dx$

  \end{edaenumerate}

\newpage


\begin{enumerate}[(1)]
  \setlength{\itemsep}{.2in}
  
  \item $\ds \int_{-1}^{5} \left( 3x^3-5x-1 \right) \ dx = \left[
      \frac{3}{4}x^4 - \frac{5}{2}x^2 - x \right]_{-1}^{5}= 402.$

  \item $u=3x-5$ とおくと,$\frac{du}{dx} = 3$ より $1=\frac{1}{3}\frac{du}{dx}$ であり,積分区間は
    \[
      \begin{array}{c|ccc}
        x & 0 & \to & -2\\ \hline
        u & -5 & \to & -11
      \end{array}
    \]
    と変換されるので,以下を得る.
    \[
      \int_{0}^{-2} \left( 3x-5 \right)^6 \ dx = \int_{0}^{-2} u^{6}\ \frac{1}{3} \frac{du}{dx} \ dx
      = \frac{1}{3} \int_{-5}^{-11} u^6 \ du = \frac{1}{3} \left[ \frac{1}{7} u^7 \right]_{-5}^{-11}
      = -\frac{6469682}{7}.
    \]
    
   \item $u=2x+1$ とおく
     と,$\frac{du}{dx}=2$ より $1=\frac{1}{2}\frac{du}{dx}$ であり,積
     分区間は
     \[
       \begin{array}{c|ccc}
         x & 1 & \to & 4\\ \hline
         u & 3 & \to & 9
       \end{array}
     \]
     と変換されるので,以下を得る.
     \[
       \int_{1}^{4} \frac{1}{(2x+1)^2} \ dx = \int_{1}^{4} \frac{1}{u^2} \ \frac{1}{2} \frac{du}{dx} \ dx
       = \frac{1}{2} \int_{3}^{9} u^{-2} \ du =\frac{1}{2} \Big[ - u^{-1} \Big]_{3}^{9} = \frac{1}{9}.
     \]

   \item $\ds \int_{0}^{\pi} \sin x \ dx =\Big[ -\cos x \Big]_{0}^{\pi}= 2.$

   \item $\ds \int_{-\pi}^{\pi} \cos x \ dx = \Big[ \sin x \Big]_{-\pi}^{\pi} =0.$

   \item $\tan x = \frac{\sin x}{\cos x}$ なので,$u=\cos x$
     とおけば $\frac{du}{dx}=-\sin x$ より $\sin x = -\frac{du}{dx}$
     であり,積分区間は
     \[
       \begin{array}{c|ccc}
         x & -\pi/6 & \to & \pi/4\\ \hline
         u & \sqrt{3}/2 & \to & \sqrt{2}/2
       \end{array}
     \]
     と変換される.よって,以下を得る.
     \begin{align*}
       \int_{-\pi/6}^{\pi/4} \tan x \ dx  &= \int_{-\pi/6}^{\pi/4} \frac{\sin x}{\cos x} \ dx =
       \int_{-\pi/6}^{\pi/4} \frac{1}{u} \left( - \frac{du}{dx} \right) \
                                            dx = -\int_{\sqrt{3}/2}^{\sqrt{2}/2} \frac{du}{u}\\
       &= -\Big[ \log |u| \Big]_{\sqrt{3}/2}^{\sqrt{2}/2}
       =\frac{1}{2} \log \frac{3}{2}.
     \end{align*}

   \item $\frac{1}{\tan x} = \frac{\cos x}{\sin x}$ なので,$u=\sin x$
     とおけば $\frac{du}{dx} = \cos x$ であり,積分区間は
     \[
       \begin{array}{c|ccc}
         x & \pi/4 & \to & \pi/3\\ \hline
         u & \sqrt{2}/2 & \to & \sqrt{3}/2
       \end{array}
     \]
     と変換される.よって,以下を得る.
     \[
       \int_{\pi/4}^{\pi/3} \frac{dx}{\tan x} = \int_{\pi/4}^{\pi/3}
       \frac{\cos x}{\sin x} \ dx = \int_{\pi/4}^{\pi/3} \frac{1}{u}
       \frac{du}{dx} \ dx = \int_{\sqrt{2}/2}^{\sqrt{3}/2}
       \frac{du}{u} = \Big[ \log|u| \Big]_{\sqrt{2}/2}^{\sqrt{3}/2}
       =\frac{1}{2}\log \frac{3}{2}.
     \]

   \item $\ds \int_{0}^{1} \sin^{-1}x \ dx = \left[ x \sin^{-1} x + \sqrt{1-x^2} \right]_{0}^{1}= \frac{\pi}{2}-1.$

   \item $\ds \int_{-1}^{1} \cos^{-1}x \ dx = \left[ x \cos^{-1} x - \sqrt{1-x^2} \right]_{-1}^{1}= \pi.$

   \item
     $\ds \int_{-\sqrt{3}}^{1} \tan^{-1}x \ dx = \left[ x \tan^{-1}x
       -\frac{1}{2} \log (x^2+1)\right]_{-\sqrt{3}}^{1} = \left(
       \frac{1}{4} - \frac{\sqrt{3}}{3}\right) \pi + \frac{1}{2} \log 2.$

   \item $\ds \int_0^1 e^x \ dx =\Big[ e^x \Big]_{0}^{1} = e-1.$

   \item
     $\ds \int_{1/2}^{3} \log x \ dx = \int_{1/2}^{3} \left( x \right)'
     \log x \ dx = \Big[ x \log x \Big]_{1/2}^{3} - \int_{1/2}^{3} dx =
     -\frac{5}{2} + \frac{1}{2} \log 2 + 3 \log 3.$

   \item $\ds \int_{0}^{\pi} x \sin x \ dx = \int_{0}^{\pi}x \left( -\cos x\right)' \ dx
     = \Big[ - x \cos x \Big]_{0}^{\pi} - \int_{0}^{\pi} \left(-\cos x\right) \ dx = \pi.$ 

   \item
     $\left(\sin x\right) \cos \left(2x\right) =
     \frac{1}{2}\left( \sin \left( 3x \right) - \sin x\right)$ なので,
     \[
       \int_{-\pi}^{\pi} \left(\sin x \right) \cos \left(2x\right) \ dx = \frac{1}{2} \left(
         \int_{-\pi}^{\pi}\sin 3x \ dx - \int_{-\pi}^{\pi} \sin x \ dx\right)
       = \frac{1}{2}\int_{-\pi}^{\pi} \sin 3x \ dx
     \]
     である.$u=3x$ とおけ
     ば $\frac{du}{dx}=3$ より $1=\frac{1}{3}\frac{du}{dx}$ であり,積分
     区間は
     \[
       \begin{array}{c|ccc}
         x & -\pi & \to & \pi\\ \hline
         u & -3\pi & \to &3 \pi
       \end{array}
     \]
     と変換される.よって,以下を得る.
     \[
       \int_{-\pi}^{\pi}\left( \sin x \right) \cos \left( 2x\right) \
       dx = \frac{1}{2} \int_{-\pi}^{\pi} \sin 3x \ dx = \frac{1}{2}
       \int_{-\pi}^{\pi}\left( \sin u \right) \frac{1}{3}
       \frac{du}{dx} \ dx = \frac{1}{6} \int_{-3\pi}^{3\pi} \sin u \
       du=0\\
     \]

   \item $\ds \int_{-1}^{1} \frac{dx}{1+x^2}= \Big[ \tan^{-1} x \Big]_{-1}^{1} = \frac{1}{2}\pi.$

   \item
     $\ds \frac{1}{4+x^2}=
     \frac{1}{4}\left(\frac{1}{1+\left(\frac{x}{2}\right)^2}\right)$ な
     ので,$u=\frac{x}{2}$ とおけ
     ば $\frac{du}{dx}=\frac{1}{2}$ より $1=2\frac{du}{dx}$ であり,積
     分区間は
     \[
       \begin{array}{c|ccc}
         x & 0 & \to & 2 \\ \hline
         u & 0 & \to & 1
       \end{array}
     \]
     と変換される.よって,以下を得る.
     \[
       \int_{0}^{4}\frac{dx}{4+x^2} = \frac{1}{4} \int_{0}^{2} \frac{dx}{1+\left(\frac{x}{2}\right)^2} = \frac{1}{4}
       \int_{0}^{2} \frac{1}{1+u^2} \ 2\  \frac{du}{dx} \ dx = \frac{1}{2} \int_{0}^{1} \frac{du}{1+u^2}
       = \frac{1}{2} \Big[ \tan^{-1} u \Big]_{0}^{1} = \frac{1}{8}\pi.
     \]


   \item 部分積分を繰り返す.
     
     \vspace{.1in}

     $\ds
     \begin{aligned}
       \int_{0}^{2} x^2 e^{-2x} \ dx &= \int_{0}^{2} x^2
       \left(-\frac{1}{2} e^{-2x} \right)' \ dx = \left[-\frac{1}{2} x^2
         e^{-2x} \right]_{0}^{2} + \int_{0}^{2} x e^{-2x} \ dx\\
       &= -2e^{-4} + \int_{0}^{2} x \left( -\frac{1}{2} e^{-2x}\right)' \ dx\\
       &= -2e^{-4} + \left[ -\frac{1}{2}x e^{-2x} \right]_{0}^{2} + \frac{1}{2} \int_{0}^{2} e^{-2x} \ dx\\
       &= -3e^{-4} + \frac{1}{2} \left[ -\frac{1}{2} e^{-2x}\right]_{0}^{2} = \frac{1}{4} - \frac{13}{4}e^{-4}.
     \end{aligned}
     $

   \item $\ds \int_{-1/2}^{1/2} \frac{dx}{\sqrt{1-x^2}} = \Big[ \sin^{-1} x \Big]_{-1/2}^{1/2} = \frac{1}{3}\pi.$

   \item $\ds \frac{1}{\sqrt{3-x^2}}
     = \frac{1}{\sqrt{ 1-\left( \frac{x}{\sqrt{3}} \right)^2} }\ \frac{1}{\sqrt{3}} $ なので,
     $u=\frac{x}{\sqrt{3}}$ とおけば $\frac{du}{dx} = \frac{1}{\sqrt{3}}$ であり,積分区間は
     \[
       \begin{array}{c|ccc}
         x & -3/2 & \to & 3/\sqrt{6}\\ \hline
         u & -\sqrt{3}/2 & \to & \sqrt{2}/2
       \end{array}
     \]
     と変換される.よって,以下を得る.
     \[
       \begin{aligned}
         \int_{-3/2}^{3/\sqrt{6}} \frac{dx}{\sqrt{3-x^2}} &
         = \int_{-3/2}^{3/\sqrt{6}} \frac{1}{\sqrt{1- \left(\frac{x}{\sqrt{3}}\right)^2 }}\
         \frac{1}{\sqrt{3}} \ dx = \int_{-3/2}^{3/\sqrt{6}} \frac{1}{\sqrt{1-u^2}}
         \frac{du}{dx}\ dx= \int_{-\sqrt{3}/2}^{\sqrt{2}/2} \frac{du}{\sqrt{1-u^2}}\\
         &= \Big[ \sin^{-1} u \Big]_{-\sqrt{3}/2}^{\sqrt{2}/2} = \frac{7}{12}\pi.
       \end{aligned}
   \]
   \item $u=x+\sqrt{x^2+2}$ とおくと,
     $\ds \frac{du}{dx}=\frac{x+\sqrt{x^2+2}}{\sqrt{x^2+2}}=\frac{u}{\sqrt{x^2+2}}$ より
     $\ds 1 = \frac{\sqrt{x^2+2}}{u} \frac{du}{dx}$ であり,積分区間は
     \[
     \begin{array}{c|ccc}
       x & 0 & \to & \sqrt{2}\\ \hline
       u & \sqrt{2} & \to & 2+\sqrt{2}
     \end{array}
   \]
   と変換されるので,以下を得る.
     \begin{align*}
       \int_{0}^{\sqrt{2}} \frac{dx}{\sqrt{x^2+2}} 
       = \int_{0}^{\sqrt{2}} \frac{1}{\sqrt{x^2+2}} \frac{\sqrt{x^2+2}}{u}\ \frac{du}{dx} \ dx
       = \int_{\sqrt{2}}^{2+\sqrt{2}} \frac{du}{u}
       = \log \left( 1+ \sqrt{2} \right).  
     \end{align*}

     あるいは,$x = \sqrt{2}~\sinh u$ とおく
     と $\frac{dx}{du} = \sqrt{2}~\cosh u$ であり,積分区間は
     \[
       \begin{array}{c|ccc}
         x & 0 & \to & \sqrt{2}\\ \hline
         u & 0 & \to & \sinh^{-1}1
       \end{array}
     \]
     と変換されるので,$\alpha=\sinh^{-1}1 \; (\Leftrightarrow \sinh \alpha =1)$ とおいて
     \[
       \int_{0}^{\sqrt{2}} \frac{dx}{\sqrt{x^2+2}} =
       \int_{0}^{\alpha}\frac{\sqrt{2}~\cosh
         u}{\sqrt{2\left(\sinh^2u+1\right)}} \ du
       = \int_{0}^{\alpha} du = \alpha= \log\left(1+\sqrt{2}\right)
     \]
     と計算することもできる.なお,$\alpha=\sinh^{-1}1$ の値は
     \[
       \alpha = \sinh^{-1} 1 \; \Leftrightarrow \; 1 = \sinh \alpha = \frac{e^{\alpha}+e^{-\alpha}}{2} \; \Leftrightarrow \;
       \left( e^{\alpha}\right)^2 - 2 e^{\alpha}  -1 =0 \; \Leftrightarrow \; e^{\alpha} = 1+\sqrt{2}
     \]
     より,$\alpha = \log\left( 1+\sqrt{2}\right)$ と求められる.
     
     
   \item $u=\sin^{-1} x \left( \Leftrightarrow x=\sin u\right)$ とおく
     と,$\frac{du}{dx} = \frac{1}{\frac{dx}{du}}=\frac{1}{\cos u}$ より
     $1=\cos u \ \frac{du}{dx}$ であり,
     積分区間は
     \[
       \begin{array}{c|ccc} x & 0 & \to & 1\\ \hline
         u & 0 & \to & \frac{\pi}{2}
       \end{array}
     \]
      と変換されるので,以下を得る.
     \[
       \int_{0}^{1}\sqrt{1-x^2} \ dx = \int_{0}^{1} \sqrt{\cos^2u} \ \cos u \ \frac{du}{dx} dx=
       \int_{0}^{\frac{\pi}{2}} \cos^2 u \ du = \frac{\pi}{4}.
     \]

   \item $\sqrt{1+4x^2} = \left( x \right)' \sqrt{1+4x^2}$ と見なして部分積分をする.
     \begin{align*}
       \int_{0}^{2} \sqrt{1+4x^2} \ dx &= \left[ x \sqrt{1+4x^2}
       \right]_{0}^{2} - \int_{0}^{2}\frac{4x^2}{\sqrt{1+4x^2}} \ dx =
       2\sqrt{17} - \int_{0}^{2} \frac{1+4x^2-1}{\sqrt{1+4x^2}} \ dx\\
       &= 2\sqrt{17} - \int_{0}^{2} \sqrt{1+4x^2} \ dx + \int_{0}^{2} \frac{dx}{\sqrt{1+4x^2}}
     \end{align*}
     より移行・整理して
     \[
       \int_{0}^{2}\sqrt{1+4x^2} \ dx  = \sqrt{17} + \frac{1}{2} \int_{0}^{2} \frac{dx}{\sqrt{1+4x^2}}
     \]
     を得る.さらに,$u=2x+\sqrt{1+4x^2}$ とけば
     \[
       \frac{du}{dx} =\frac{4x+2\sqrt{1+4x^2}}{\sqrt{1+4x^2}} =
       \frac{2u}{\sqrt{1+4x^2}}
     \]
     より $\ds 1 = \frac{\sqrt{1+4x^2}}{2u} \ \frac{du}{dx}$ であり,積分区
     間は
     \[
       \begin{array}{c|ccc} x & 0 & \to & 2\\ \hline
         u & 1 & \to & 4+\sqrt{17}
       \end{array}
     \]
     と変換される.よって,以下を得る.
     \begin{align*}
       \int_{0}^{4} \sqrt{1+4x^2} \ dx = \sqrt{17} + \frac{1}{2} \int_{1}^{4+\sqrt{17}} \frac{du}{2u}
       = \sqrt{17} + \frac{1}{4} \log \left( 4+\sqrt{17} \right).
     \end{align*}

     あるいは,$x=\frac{1}{2}\sinh u$ とおけば $\frac{dx}{du} = \frac{1}{2} \cosh u$ であり,積分区間は
     \[
       \begin{array}{c|ccc}
         x & 0 & \to & 2\\ \hline
         u & 0 & \to & \sinh^{-1}4
       \end{array}
     \]
     と変換されるので,$\alpha=\sinh^{-1}4 \;(\Leftrightarrow \sinh \alpha =4)$ とおいて
     \[
       \begin{aligned}
         \int_{0}^{2}\sqrt{1+4x^2} \ dx &= \int_{0}^{\alpha} \sqrt{1+\sinh^2 u}\  \frac{1}{2}\cosh u \ du
         =\frac{1}{2}\int_{0}^{\alpha} \cosh^2 u \ du\\
         &= \frac{1}{2}\int_{0}^{\alpha} \frac{1+\cosh (2u)}{2} \ du
         = \frac{1}{4} \left[ u + \frac{1}{2}\sinh(2u)\right]_{0}^{\alpha}\\
         &= \frac{\alpha}{4} + \frac{\sinh(2\alpha)}{8}
         =\frac{\alpha}{4} + \frac{(\sinh \alpha) \cosh \alpha}{4}
         =\frac{\alpha}{4} + \frac{(\sinh \alpha) \sqrt{1+\sinh^2\alpha}}{4}\\
         &= \frac{1}{4}\log\left(4+\sqrt{17}\right) + \sqrt{17}
       \end{aligned}
     \]
     と計算することもできる.なお,$\alpha=\sinh^{-1}4$ の値は
     \[
       \alpha = \sinh^{-1}4 \; \Leftrightarrow \; 4 = \sinh \alpha = \frac{e^{\alpha}-e^{-\alpha}}{2}
       \; \Leftrightarrow \; \left( e^{\alpha}\right)^2 - 8 e^{\alpha} -1=0 \; \Leftrightarrow \;
       e^{\alpha} = 4 + \sqrt{17}
     \]
     より.$\alpha = \log\left(4+\sqrt{17}\right)$ と求められる.


   \item $\ds 3^x  = e^{(\log 3) x}$ なので,$u=(\log 3) x$ とおけば $\frac{du}{dx}=\log 3$ より
     $1 = \frac{1}{\log 3} \frac{du}{dx}$ であり,積分区間は
     \[
       \begin{array}{c|ccc}
         x & 0 & \to & 1\\ \hline
         u & 0 & \to & \log 3
       \end{array}
     \]
     と変換される.よって,以下を得る.
     \[
       \int_{0}^{1} 3^x \ dx = \int_{0}^{1} e^{(\log 3)x}  dx = \int_{0}^{1}
       \frac{e^u}{\log 3} \frac{du}{dx} \ dx =\frac{1}{\log 3} \int_{0}^{\log
         3} e^{u} \ du = \frac{1}{\log 3} \Big[ e^{u} \Big]_{0}^{\log 3} =
       \frac{2}{\log 3}.
     \]

   \item $\ds \int_{1}^{2} \log_2 x \ dx = \int_{1}^{2} \frac{\log x}{\log 2}
     \ dx = \frac{1}{\log 2} \Big[ x \log x - x\Big]_{1}^{2} = 2 -\frac{1}{\log 2}.$

   \item $\ds \frac{x^3}{\sqrt{x^2+5}} =  x^2 \cdot \frac{x}{\sqrt{x^2+5}}$ なので,
     $u=\sqrt{x^2+5}$ とおけば $\frac{du}{dx}=\frac{x}{\sqrt{x^2+5}}$ であり,積分区間は
     \[
       \begin{array}{c|ccc}
         x & 0 & \to & 1 \\ \hline
         u & \sqrt{5} & \to & \sqrt{6}
       \end{array}
     \]
     と変換される.よって,以下を得る.
     \[
       \begin{aligned}
         \int_{0}^{1}\frac{x^3}{\sqrt{x^2+5}} \ dx &=\int_{0}^{1} x^2 \cdot \frac{x}{\sqrt{x^2+5}} \ dx
           = \int_{0}^{1} (u^2-5) \ \frac{du}{dx} \ dx
         =\int_{\sqrt{5}}^{\sqrt{6}} (u^2-5) \ du \\
         &= \Big[ \frac{1}{3}u^3 - 5u \Big]_{\sqrt{5}}^{\sqrt{6}}
         = -3\sqrt{6} + \frac{10}{3}\sqrt{5}.
       \end{aligned}
     \]

   \item $\left( \log x\right)^2 = \left( x \right)' \left( \log x\right)^2$ と見なして部分積分をする.
     \begin{align*}
       \int_1^e \left( \log x \right)^2 \ dx 
       & = \Big[ x \left(\log x\right)^2 \Big]_{1}^{e} - \int_{1}^{e} x \cdot \frac{2 \log x}{x} \ dx
        =e - 2\int_{1}^{e} \log x \ dx\\
       & = e- 2\Big[ x \log x - x \Big]_{1}^{e}= e-2.
     \end{align*}

     

   \item $\ds \frac{1}{x^2+1} = \left(x\right)' \frac{1}{x^2+1}$ と見なし
     て,$\ds \int \frac{dx}{x^2+1}$ に部分積分を適用して
     \[
       \begin{aligned}
         \int \frac{dx}{x^2+1} &= \frac{x}{x^2+1} + 2\int \frac{x^2}{(x^2+1)^2} \ dx
         = \frac{x}{x^2+1} +2 \int \frac{x^2+1 -1}{(x^2+1)^2} \ dx\\
         &= \frac{x}{x^2+1} + 2 \left(\int \frac{dx}{x^2+1} - \int \frac{dx}{(x^2+1)^2} \right)
       \end{aligned}
     \]
     を得る.上式を移項・整理して
     \[
       \int \frac{dx}{(x^2+1)^2} = \frac{1}{2} \left( \frac{x}{x^2+1} + \int \frac{dx}{x^2+1} \right)
     \]
     を得る.よって,以下を得る.
     \[
       \int_{0}^{1} \frac{dx}{(x^2+1)^2} = \frac{1}{2}
       \left[\frac{x}{x^2+1}\right]_{0}^{1} +\frac{1}{2} \int_{0}^{1}
       \frac{dx}{x^2+1} = \frac{1}{4} + \frac{1}{2} \Big[ \tan^{-1} x
       \Big]_{0}^{1} = \frac{1}{4} + \frac{\pi}{8}.
     \]



   \item 被積分関数の分母を平方完成すると,
     \[
       x^2+x+1 = \left(x + \frac{1}{2} \right)^2 + \frac{3}{4} 
       = \frac{3}{4}\left(\left(\frac{2}{\sqrt{3}}\left(x+\frac{1}{2}\right)\right)^2 +1\right)
     \]
     である.ここで,$u=\frac{2}{\sqrt{3}} \left(x+\frac{1}{2}\right)$ とおく
     と $\frac{du}{dx} = \frac{2}{\sqrt{3}}$ より
     $1 = \frac{\sqrt{3}}{2} \frac{du}{dx}$ であり,積分区間は
     \[
       \begin{array}{c|ccc}
         x & 0 & \to & 1\\ \hline
         u & 1/\sqrt{3} & \to & \sqrt{3}
       \end{array}
     \]
     と変換される.よって,以下を得る.
     \begin{align*}
      \int_{0}^{1} \frac{dx}{x^2+x+1} 
       &=  \frac{4}{3} \int_{0}^{1} \frac{1}{\frac{4}{3}\left( x+\frac{1}{2}\right)^2  +1}  \ dx
       =\frac{4}{3} \int_{0}^{1} \frac{1}{u^2+1} \frac{\sqrt{3}}{2} \frac{du}{dx} \ dx\\
       &  = \frac{2}{\sqrt{3}} \int_{1/\sqrt{3}}^{\sqrt{3}} \frac{du}{u^2+1} 
       = \frac{2}{\sqrt{3}} \Big[ \tan^{-1} u \Big]_{1/\sqrt{3}}^{\sqrt{3}} = \frac{\pi}{9}\sqrt{3}.
     \end{align*}

   \item
     $\ds \int_{-1}^{1} \frac{dx}{x^2-4} =\frac{1}{4} \int_{-1}^{1}
     \left( \frac{1}{x-2} - \frac{1}{x+2}\right) \ dx=\frac{1}{4}\Big[
     \log|x-2| - \log |x+2| \Big]_{-1}^{1} = -\frac{1}{2} \log 3.$

   \item $\ds \int_{0}^{1} \frac{x+2}{(x^2+x+1)^2} \ dx 
     = \frac{1}{2} \int_{0}^{1} \frac{2x+1}{(x^2+x+1)^2} \ dx + \frac{3}{2} \int_{0}^{1} \frac{dx}{(x^2+x+1)^2}$ である.

     \vspace{.1in}

     \begin{itemize}
       \setlength{\itemsep}{.1in}
       
     \item 右辺第1項について:
       \[
         \frac{1}{2}\int_{0}^{1}\frac{2x+1}{(x^2+x+1)^2} \ dx =\frac{1}{2}\int_{0}^{1}
       \frac{(x^2+x+1)'}{(x^2+x+1)^2} \ dx = \frac{1}{2}\left[ -\frac{1}{x^2+x+1}
       \right]_{0}^{1} = \frac{1}{3}.
     \]
       
     \item 右辺第2項について:
       \[
         \left(x^2+x+1\right)^2 
         = \left( \frac{3}{4}\left( \left(\frac{2}{\sqrt{3}}\left( x+ \frac{1}{2} \right) \right)^2 + 1 \right) \right)^2
       \]
       なので,$u= \frac{2}{\sqrt{3}}\left(x+\frac{1}{2}\right)$ と
       おくと $\frac{du}{dx} =
       \frac{2}{\sqrt{3}}$ より $1=\frac{\sqrt{3}}{2} \frac{du}{dx}$ で
       あり,積分区間は
       \[
         \begin{array}{c|ccc}
           x & 0 & \to & 1 \\ \hline
           u & 1/\sqrt{3} & \to & \sqrt{3}
         \end{array}
       \]
       と変換される.よって,以下を得る.
       \[
         \begin{aligned}
           \frac{3}{2} \int_{0}^{1} \frac{dx}{\left( x^2+x+1\right)^2} &=
           \frac{8}{3} \int_{0}^{1} \frac{1}{\left( \frac{4}{3} \left( x + \frac{1}{2}\right)^2 + 1\right)^2} \ dx
           = \frac{8}{3} \int_{0}^{1} \frac{1}{(u^2+1)^2} \frac{\sqrt{3}}{2} \frac{du}{dx} \ dx\\
           &=\frac{4}{\sqrt{3}} \int_{1/\sqrt{3}}^{\sqrt{3}} \frac{dv}{(u^2+1)^2}
           = \frac{4}{\sqrt{3}} \left[ \frac{1}{2}\frac{u}{u^2+1} 
             + \frac{1}{2} \tan^{-1} u \right]_{1/\sqrt{3}}^{\sqrt{3}} \\
           &= \frac{\pi}{9}\sqrt{3}.
         \end{aligned}
       \]
     \end{itemize}
     以上より,$\ds \int_{0}^{1} \frac{dx}{(x^2+x+1)^2} = \frac{1}{3} + \frac{\pi}{9}\sqrt{3}$ である.


   \item 被積分関数を部分分数に分解する.
     \begin{align*}
       \int_{1}^{2} \frac{dx}{x^2(x+1)} 
       &= \int_{1}^{2} \left(-\frac{1}{x} + \frac{1}{x^2} + \frac{1}{x+1}\right) \ dx
       = \left[ - \log|x| -\frac{1}{x} + \log|x+1|\right]_{1}^{2}\\
       &= \frac{1}{2} + \log 3 - 2 \log 2.
     \end{align*}

         

   \item
     $\ds \frac{1+x^2}{1-x^2} = \frac{-(1-x^2)+2}{1-x^2} = -1 +
     \frac{2}{1-x^2} = -1 +\frac{1}{1-x}+\frac{1}{1+x}$ より,以下を得る.
     \[
       \int_{0}^{\frac{1}{2}} \frac{1+x^2}{1-x^2} \ dx 
         = \Big[ -x - \log\left|1-x\right| + \log\left|1+x\right|\Big]_{0}^{\frac{1}{2}}
         = \log 3 - \frac{1}{2}.
     \]

   \item 被積分関数の分母を因数分解し,部分分数に分解して 
     \begin{align*}
       \int_{0}^{1} \frac{dx}{x^4+1} 
       &= \int_{0}^{1}
         \frac{dx}{(x^2+\sqrt{2}x+1)(x^2-\sqrt{2}x+1)}\\
       &= \frac{1}{4} \int_{0}^{1} \frac{\sqrt{2}x+2}{x^2+\sqrt{2}x+1}
         \ dx
         - \frac{1}{4} \int_{0}^{1} \frac{\sqrt{2}x-2}{x^2-\sqrt{2}x+1} \ dx\\
       &= \frac{1}{4} \left( \frac{1}{\sqrt{2}} \int_{0}^{1} \frac{2x+\sqrt{2}x}{x^2+\sqrt{2}x+1} \ dx 
         + 2 \int_{0}^{1} \frac{dx}{ \left(\sqrt{2}x+1 \right)^2+1} \right)\\
       & \quad - \frac{1}{4} \left( \frac{1}{\sqrt{2}} \int_{0}^{1} \frac{2x-\sqrt{2}}{x^2-\sqrt{2}x+1} \ dx
         -2 \int_{0}^{1} \frac{dx}{\left(\sqrt{2}x-1\right)^2+1} \right)\\
       & = \frac{1}{4\sqrt{2}} \left[ \log \left( x^2+\sqrt{2}x+1 \right) \right]_{0}^{1}
         + \frac{1}{2} \int_{0}^{1} \frac{dx}{\left( \sqrt{2}x+1\right)^2+1}\\
       & \quad - \frac{1}{4\sqrt{2}} \left[ \log \left( x^2-\sqrt{2}x+1\right) \right]_{0}^{1}
         +\frac{1}{2} \int_{0}^{1} \frac{dx}{\left(\sqrt{2}x-1\right)^2+1}\\
       &= \frac{\sqrt{2}}{8} \log \left( 3+2\sqrt{2}\right)  
         + \frac{1}{2}\left(\int_{0}^{1} \frac{dx}{\left(\sqrt{2}x+1\right)^2+1}
         + \int_{0}^{1}\frac{dx}{\left( \sqrt{2}x-1\right)^2 +1} \right)
     \end{align*}
     を得る.$u=\sqrt{2}x+1, \, v=\sqrt{2}x-1$
     とおくと,$\frac{du}{dx} = \sqrt{2}, \, \frac{dv}{dx}=\sqrt{2}$ であ
     り,積分区間は
     \[
       \begin{array}{c|rcc}
         x & 0 & \to & 1\\ \hline
         u & 1 & \to & \sqrt{2}+1 
       \end{array} \quad
       \begin{array}{c|ccc}
         x & 0 & \to & 1 \\ \hline
         v & -1 & \to & \sqrt{2}-1
       \end{array}
     \]
     と変換される.よって,以下を得る.     
     \[
       \begin{aligned}
         \int_{0}^{1} \frac{dx}{x^4+1} &= \frac{\sqrt{2}}{8} \log \left( 3+2\sqrt{2} \right)
         + \frac{1}{2} \left( \int_{0}^{1} \frac{1}{u^2+1} \frac{1}{\sqrt{2}} \frac{du}{dx} \ dx 
           + \int_{0}^{1} \frac{1}{v^2+1} \frac{1}{\sqrt{2}} \frac{dv}{dx} \ dx \right)\\
         &= \frac{\sqrt{2}}{8} \log \left(3+2\sqrt{2}\right) + \frac{\sqrt{2}}{4}
         \left( \int_{1}^{\sqrt{2}+1} \frac{du}{u^2+1} + \int_{-1}^{\sqrt{2}-1} \frac{dv}{v^2+1}\right)\\
         &= \frac{\sqrt{2}}{8} \log \left( 3 + 2\sqrt{2} \right) + \frac{\sqrt{2}}{4} \left(
           \Big[ \tan^{-1} u \Big]_{1}^{\sqrt{2}+1} + \Big[ \tan^{-1} v \Big]_{-1}^{\sqrt{2}-1}\right)\\
         &= \frac{\sqrt{2}}{8} \log \left( 3 +2 \sqrt{2} \right) + \frac{\sqrt{2}}{4} \left(
           \tan^{-1}\left( \sqrt{2}+1 \right) + \tan^{-1} \left( \sqrt{2}-1 \right) \right)\\
         &= \frac{\sqrt{2}}{8} \log \left( 3+2\sqrt{2} \right) + \frac{\sqrt{2}}{4} \left(
           \tan^{-1} \left(\sqrt{2}+1\right) + \tan^{-1} \frac{1}{\sqrt{2}+1} \right)\\
         &= \frac{\sqrt{2}}{8} \log \left(3+2\sqrt{2}\right) + \frac{\pi}{8}\sqrt{2}.
       \end{aligned}
     \]
     ただし,途中で $x \in \left(0,
       \frac{\pi}{2}\right)$ に対して成り立つ以下の関係式を用いた.
     \[
       \tan^{-1}x + \tan^{-1}\frac{1}{x} = \frac{\pi}{2}
     \]


   \item $u=\sqrt{x}$ とおく
     と,$\frac{du}{dx}=\frac{1}{2\sqrt{x}}=\frac{1}{2u}$
     より$1= 2u \ \frac{du}{dx}$ であり,積分区間は
     \[
       \begin{array}{c|ccc}
         x & 1 & \to & 4 \\ \hline
         u & 1 & \to & 2
       \end{array}
     \]
     と変換される.よって,以下を得る.
     \[
       \begin{aligned}
         \int_{1}^{4}\frac{1}{\sqrt{x}+1} \ dx &= \int_{1}^{4} \frac{1}{u+1} \ 2u \ \frac{du}{dx} \ dx
         = 2 \int_{1}^{2} \frac{u}{u+1} \ du = 2 \int_{1}^{2} \frac{u+1-1}{u+1} \ du\\
         &= 2 \int_{1}^{2} \left( 1 - \frac{1}{u+1} \right) \ du = 2 \Big[ u - \log|u+1| \Big]_{1}^{2}
         = 2 + 2 \log\frac{2}{3}.
       \end{aligned}
     \]

   \item $\ds \int_{1}^{\sqrt{2}} \frac{dx}{x^3+x} = \int_{1}^{\sqrt{2}} \left( \frac{1}{x} - \frac{x}{x^2+1}\right) \ dx
     = \left[ \log |x| - \frac{1}{2} \log \left(x^2+1 \right)
     \right]_{1}^{\sqrt{2}} = \log \frac{2}{\sqrt{3}}. $
     
   \item $\ds \int_{0}^{1} \frac{dx}{\sqrt{x}} = \lim_{a \to +0} \int_{a}^{1}x^{-\frac{1}{2}} \ dx
     = \lim_{a \to +0} \Big[ 2x^{\frac{1}{2}} \Big]_{a}^{1} = \lim_{a \to +0}\left(2-2\sqrt{a}\right)=2.$

   \item $\ds \int_{1}^{\infty}\frac{dx}{x^3} = \lim_{b \to \infty} \int_{1}^{b} x^{-3} \ dx
     = \lim_{b \to \infty} \Big[ -\frac{1}{2}x^{-2}\Big]_{1}^{b} 
     = \lim_{b \to \infty} \frac{1}{2}\left( 1- \frac{1}{b^2}\right) = \frac{1}{2}.$

   \item $\ds \int_{0}^{2} \frac{dx}{\sqrt{4-x^2}} = \lim_{b \to 2-0} \int_{0}^{b} \frac{dx}{\sqrt{4-x^2}}$ である.

     \vspace{1zh}
     
     $u=\frac{x}{2}$ とおくと,$\frac{du}{dx}=\frac{1}{2}$ より $1=2\ \frac{du}{dx}$ であり,積分区間は
     \[
       \begin{array}{c|ccc}
         x & 0 & \to & b\\ \hline
         u & 0 & \to & \frac{b}{2}
       \end{array}
     \]
     と変換される.これより,
     \[
       \int_{0}^{b} \frac{dx}{\sqrt{4-x^2}} = \int_{0}^{b} \frac{1}{2\sqrt{1-u^2}}\  2\ \frac{du}{dx} \ dx
       = \int_{0}^{\frac{b}{2}} \frac{du}{\sqrt{1-u^2}}
       =\Big[ \sin^{-1} u \Big]_{0}^{\frac{b}{2}} = \sin^{-1} \frac{b}{2}
     \]
     なので,以下を得る.
     \[
       \int_{0}^{2} \frac{dx}{\sqrt{4-x^2}} = \lim_{b \to 2-0} \sin^{-1} \frac{b}{2} = \sin^{-1} 1 = \frac{\pi}{2}.
     \]

   \item $\ds \int_{1}^{2} \frac{dx}{x\sqrt{x-1}} = \lim_{a \to 1+0} \int_{a}^{2} \frac{dx}{x\sqrt{x-1}}$ である.

     \vspace{1zh}
     
     $u=\sqrt{x-1}$ とおくと $x=u^2+1$ である.ま
     た,$\frac{du}{dx}=\frac{1}{2\sqrt{x-1}}$ より $\frac{1}{\sqrt{x-1}} = 2 \frac{du}{dx}$ であり,積分区間は
     \[
       \begin{array}{c|ccc}
         x & a & \to & 2 \\ \hline
         u & \sqrt{a-1} & \to & 1
       \end{array}
     \]
     と変換される.これより,
     \begin{align*}
       \int_{a}^{2} \frac{dx}{x\sqrt{x-1}} 
       &= \int_{a}^{2} \frac{1}{(u^2+1)} \ 2 \frac{du}{dx} \ dx 
         = 2\int_{\sqrt{a-1}}^{1} \frac{du}{u^2+1} = 2 \Big[ \tan^{-1} u \Big]_{\sqrt{a-1}}^{1} \\
       &= \frac{\pi}{2} - 2 \tan^{-1} \sqrt{a-1}
     \end{align*}
     なので,以下を得る.
     \[
       \int_{1}^{2} \frac{dx}{x\sqrt{x-1}} = \lim_{a \to 1+0} \left( \frac{\pi}{2} - 2 \tan^{-1}\sqrt{a-1} \right)
       = \frac{\pi}{2}.
     \]

   \item $\ds \int_{-1}^{1} \frac{dx}{\sqrt[4]{(1+x)^3}} 
     = \lim_{a \to -1+0} \int_{a}^{0} \left(1+x \right)^{-\frac{3}{4}} \ dx
     + \lim_{b \to 1-0} \int_{0}^{b} \left( 1+x \right)^{-\frac{3}{4}} \ dx$ である.
     \begin{align*}
        &\int_{a}^{0} (1+x)^{-\frac{3}{4}} \ dx = \left[ 4 (1+x)^{\frac{1}{4}} \right]_{a}^{0}
          = 4 - 4 \left( 1+a\right)^{\frac{1}{4}} \to 4 \; ( a \to -1+0),\\
        &\int_{0}^{b} (1+x)^{-\frac{3}{4}} \ dx = \left[ 4 (1+x)^{\frac{1}{4}} \right]_{0}^{b}
       = 4\left( 1+b\right)^{\frac{1}{4}} - 4 \to 4 \cdot 2^{\frac{1}{4}}-4 \; (b \to 1-0)
     \end{align*}
     だから,以下を得る.
     \[
       \int_{-1}^{1} \frac{dx}{\sqrt[4]{(1+x)^3}} = 4+ 4 \cdot 2^{\frac{1}{4}} - 4 = 4 \sqrt[4]{2}.
     \]

   \item
     $\ds \int_{-2}^{2} \frac{x}{\sqrt{4-x^2}} \ dx = \lim_{a \to
       -2+0} \int_{a}^{0} \frac{x}{\sqrt{4-x^2}} \ dx + \lim_{b \to
       2-0} \int_{0}^{b} \frac{x}{\sqrt{4-x^2}} \ dx$ である.

     \vspace{1zh}
     
     $\ds u=\sqrt{4-x^2}$ とおく
     と,$\ds \frac{du}{dx}=-\frac{x}{\sqrt{4-x^2}}$ よ
     り $\ds \frac{x}{\sqrt{4-x^2}} = -\frac{du}{dx}$ であるから,
     \[
       \int \frac{x}{\sqrt{4-x^2}} = \int -\frac{du}{dx} \ dx = -\int du = -u+C = -\sqrt{4-x^2}+C
     \]
     である.ここで,$C$ は任意の実数である.これより,以下を得る.
     \begin{align*}
       \int_{-2}^{2} \frac{x}{\sqrt{4-x^2}} \ dx 
       &= \lim_{a \to -2+0} \left[ -\sqrt{4-x^2} \right]_{a}^{0} 
         + \lim_{b \to 2-0} \left[ -\sqrt{4-x^2} \right]_{0}^{b}\\
       &= \lim_{a \to -2+0} \left( -2 +\sqrt{4-a^2}\right)
         + \lim_{b \to 2-0} \left( -\sqrt{4-b^2} +2\right)\\
       &= -2 + 2 = 0.
     \end{align*}

   \item $\ds \int_{0}^{\infty} x e^{-x^2} \ dx = \lim_{b \to \infty} \int_{0}^{b} x e^{-x^2} \ dx$ である.

     \vspace{1zh}

     $u= e^{-x^2}$ とおくと,$\frac{du}{dx} = -2x e^{-x^2}$
     より $x e^{-x^2} = -\frac{1}{2} \frac{du}{dx}$ であり,積分区間は
     \[
       \begin{array}{c|ccc}
         x & 0 & \to & b\\ \hline
         u & 1 & \to & e^{-b^2}
       \end{array}
     \]
     と変換される.よって,
     \[
       \int_{0}^{b} x e^{-x^2} \ dx = \int_{0}^{b} -\frac{1}{2} \frac{du}{dx} \ dx = - \frac{1}{2}\int_{1}^{e^{-b^2}} \ du
       = \frac{1}{2} \left(1 - e^{-b^2}\right)
     \]
     より,以下を得る.
     \[
       \int_{0}^{\infty} x e^{-x^2} \ dx = \lim_{b \to \infty} \frac{1}{2} \left( 1 - e^{-b^2} \right) = \frac{1}{2}.
     \]

   \item $\ds \int_{0}^{1} x \log x \ dx = \lim_{a \to +0} \int_{a}^{1} x  \log x \ dx$ である.
     \[
       \int_{a}^{1} x \log x \ dx = \left[ \frac{1}{2}x^2 \log x \right]_{a}^{1} - \int_{a}^{1} \frac{1}{2} x \ dx
       = - \frac{1}{2}a^2 \log a - \frac{1}{4} \left(1-a^2\right)
     \]
     より,
     \[
       \int_{0}^{1} x \log x \ dx = \lim_{a \to +0} \left(-\frac{1}{2}a^2 \log a - \frac{1}{4}\left( 1-a^2 \right) \right)
                                    = -\frac{1}{2} \lim_{a \to +0}  \frac{\log a}{\frac{1}{a^2}} - \frac{1}{4}
     \]
     を得る.さらに,ここで
     \[
       \lim_{a \to +0} \frac{ \left( \log a\right)'}{\left( \frac{1}{a^2} \right)'} 
       = \lim_{a \to +0} \frac{\frac{1}{a}}{-\frac{2}{a^3}} = \lim_{a \to +0} -\frac{1}{2} a^2 =0
     \]
     より,ロピタルの定理から以下を得る.
     \[
       \int_{0}^{1} x \log x \ dx = -\frac{1}{2} \lim_{a \to +0} \frac{\log a}{\frac{1}{a^2}} - \frac{1}{4} = -\frac{1}{4}.
     \]

   \item $\ds \int_{-\infty}^{\infty} \frac{dx}{x^2+4} = \lim_{a \to -\infty} \int_{a}^{0} \frac{dx}{x^2+4}
     + \lim_{b \to \infty} \int_{0}^{b}\frac{dx}{x^2+4}$ である.

     \vspace{1zh}
     
     $u=\frac{x}{2}$ とおくと,$\frac{du}{dx}=\frac{1}{2}$ より $1 = 2\ \frac{du}{dx}$ であるから,
     \[
       \int \frac{dx}{x^2+4} = \int \frac{1}{4(u^2+1)} \ 2\ \frac{du}{dx} \ dx = \frac{1}{2}\int \frac{du}{u^2+1}
       = \frac{1}{2} \tan^{-1} u +C = \frac{1}{2} \tan^{-1}\frac{x}{2} +C
     \]
     を得る.ここで,$C$ は任意の実数である.これより,以下を得る.
     \begin{align*}
       \int_{-\infty}^{\infty} \frac{dx}{x^2+4} 
       &= \lim_{a \to -\infty} \frac{1}{2} \left[ \tan^{-1} \frac{x}{2} \right]_{a}^{0}
         + \lim_{b \to \infty} \frac{1}{2} \left[ \tan_{-1} \frac{x}{2} \right]_{0}^{b}\\
       &= \frac{1}{2} \lim_{a \to -\infty} \left(-\tan^{-1}\frac{a}{2}\right)
         + \frac{1}{2} \lim_{b \to \infty} \tan^{-1} \frac{b}{2}=\frac{\pi}{4} + \frac{\pi}{4} = \frac{\pi}{2}.
     \end{align*}

   \item $\ds \int_{1}^{\infty}\frac{\log x}{x^2} \ dx = \lim_{b \to \infty} \int_{1}^{b} \frac{\log x}{x^2} \ dx$ である.
     \[
       \int_{1}^{b} \frac{\log x}{x^2} \ dx = \left[ -\frac{1}{x} \log x \right]_{1}^{b} + \int_{1}^{b} \frac{1}{x^2} \ dx
       = -\frac{\log b}{b} + \left[-\frac{1}{x}\right]_{1}^{b} = -\frac{\log b}{b} -\frac{1}{b} +1
     \]
     であるが,ここで
     \[
       \lim_{ b \to \infty} \frac{\left( \log b\right)'}{b'} = \lim_{b \to \infty} \frac{1}{b} = 0
     \]
     より,ロピタルの定理から以下を得る.
     \[
       \int_{1}^{\infty} \frac{\log x}{x^2} \ dx  = \lim_{b \to \infty} \left( -\frac{\log b}{b} - \frac{1}{b} +1\right) = 1.
     \]
     

   \item
     $\ds \int_{0}^{\infty} \frac{dx}{e^x (1+e^x)} = \lim_{b \to
       \infty} \int_{0}^{b} \frac{dx}{e^x\left(1+e^x\right)}$ であ
     る.

     \vspace{1zh}

     $u=e^{x}$ とおくと,$\frac{du}{dx}=e^{x}=u$ より
     $1 =\frac{1}{u}\frac{du}{dx}$ であり,積分区間は
     \[
       \begin{array}{c|ccc}
         x & 0 & \to & b\\ \hline
         u & 1 & \to & e^{b}
       \end{array}
     \]
     と変換される.これより
     \begin{align*}
       \int_{0}^{b} \frac{dx}{e^x(1+e^x)} 
       &= \int_{0}^{b} \frac{1}{u(1+u)} \frac{1}{u} \frac{du}{dx} \ dx
         = \int_{1}^{e^{b}} \frac{du}{u^2(1+u)}\\
       &= \int_{1}^{e^{b}} \left( \frac{1}{u^2} - \frac{1}{u} + \frac{1}{1+u} \right) \ du
         = \left[ -\frac{1}{u} - \log|u| + \log |1+u| \right]_{1}^{e^{b}}\\
       &=1- \log 2 - e^{-b} + \log \left(1+e^{-b}\right)
     \end{align*}
     なので,以下を得る.
     \[
       \int_{0}^{\infty} \frac{dx}{e^x (1+e^x)} = \lim_{b \to \infty} \left(1 - \log 2 -e^{-b} 
         + \log \left(1+e^{-b}\right) \right)= 1 - \log 2.
     \]
     
   \item $\ds \int_{0}^{\infty}\frac{dx}{(x^2+1)(x^2+4)}= \lim_{b \to \infty} \int_{0}^{b} \frac{dx}{(x^2+1)(x^2+4)}$ である.
     \begin{align*}
       \int_{0}^{b}\frac{dx}{(x^2+1)(x^2+4)} 
       &= \frac{1}{3}
         \int_{0}^{b} \frac{dx}{x^2+1}- \frac{1}{3} \int_{0}^{b}
         \frac{dx}{x^2+4} = \frac{1}{3} \Big[ \tan^{-1} x \Big]_{0}^{b}
         - \frac{1}{12} \int_{0}^{b} \frac{dx}{\frac{x^2}{4}+1}\\
       &= \frac{1}{3} \tan^{-1} b - \frac{1}{12} \int_{0}^{b} \frac{dx}{\frac{x^2}{4}+1}
     \end{align*}
     なので,$u=\frac{x}{2}$ とおくと $\frac{du}{dx} =
     \frac{1}{2}$ より $1 = 2 \ \frac{du}{dx}$ であり,積分区間は
     \[
       \begin{array}{c|ccc}
         x & 0 & \to & b\\ \hline
         u & 0 & \to & \frac{b}{2}
       \end{array}
     \]
     と変換される.これより
     \begin{align*}
       \int_{0}^{b} \frac{dx}{(x^2+1)(x^2+4)} 
       &= \frac{1}{3} \tan^{-1}b -
         \frac{1}{12} \int_{0}^{b} \frac{1}{u^2+1} \ 2\ \frac{du}{dx} \ dx
       =\frac{1}{3} \tan^{-1} b - \frac{1}{6} \int_{0}^{\frac{b}{2}} \frac{du}{u^2+1}\\
       &= \frac{1}{3} \tan^{-1} b - \frac{1}{6} \Big[\tan^{-1} u \Big]_{0}^{\frac{b}{2}}
        = \frac{1}{3} \tan^{-1} b - \frac{1}{6} \tan^{-1}\frac{b}{2}
     \end{align*}
     だから,以下を得る.
     \[
       \int_{0}^{\infty} \frac{dx}{(x^2+1)(x^2+4)} = \lim_{b \to \infty}
       \left( \frac{1}{3} \tan^{-1} b - \frac{1}{6} \tan^{-1} \frac{b}{2}
       \right) = \frac{\pi}{12}.
     \]

   \item
     $\ds \int_{0}^{1} \frac{x^4}{\sqrt{1-x^2}} \ dx = \lim_{b \to
       1-0}\int_{0}^{b} \frac{x^4}{\sqrt{1-x^2}} \ dx$ であ
     る.

     \vspace{1zh}

     $u=\sin^{-1}x \left( \Leftrightarrow x=\sin u\right)$ とおく
     と,$\ds \frac{du}{dx} = \frac{1}{\sqrt{1-x^2}}$ であり,積分区間は
     \[
       \begin{array}{c|ccc}
         x & 0 & \to & b\\ \hline
         u & 0 & \to & \sin^{-1}b
       \end{array}
     \]
     と変換される.これより
     \begin{align*}
       \int_{0}^{b} \frac{x^4}{\sqrt{1-x^2}} \ dx 
       &= \int_{0}^{\sin^{-1}b} \sin^4 u \ du
       =\int_{0}^{\sin^{-1}b} \left( \frac{1+\cos 2u}{2}\right)^2 \ du\\
       & = \frac{1}{4}\int_{0}^{\sin^{-1}b} \left( 1+ 2 \cos 2u + \cos^2 2u\right) \ du\\
       &=\frac{1}{4} \left( \Big[ x + \sin 2u \Big]_{0}^{\sin^{-1}b} 
         + \int_{0}^{\sin^{-1}b} \frac{1+\cos 4u}{2} \ du\right)\\
       &= \frac{1}{4} \left( \sin^{-1}b + \sin \left(2 \sin^{-1}b\right)
         + \frac{1}{2}\Big[ u+\frac{1}{4}\sin 4u\Big]_{0}^{\sin^{-1}b} \right)\\
       &= \frac{3}{8} \sin^{-1}b + \frac{1}{4} \sin\left(2\sin^{-1}b\right) + \frac{1}{8}\sin\left(4\sin^{-1}b\right)
     \end{align*}
     なので,以下を得る.
     \begin{align*}
       \int_{0}^{1} \frac{x^4}{\sqrt{1-x^2}} \ dx 
       &= \lim_{b \to 1-0}\left( \frac{3}{8}\sin^{-1}b + \frac{1}{4} \sin\left(2\sin^{-1}b\right) 
       + \frac{1}{8}\sin\left(4\sin^{-1}b\right) \right)\\
       &=\frac{3}{8} \sin^{-1}1 + \frac{1}{4} \sin\left(2 \sin^{-1}1\right) + \frac{1}{8}\sin\left(4\sin^{-1}1\right)\\
       &= \frac{3}{16}\pi + \frac{1}{4}\sin \pi + \frac{1}{8}\sin 2\pi=\frac{3}{16}\pi.
     \end{align*}
     
   \end{enumerate}


\end{document}
